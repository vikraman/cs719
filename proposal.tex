\documentclass{article}

\usepackage{authblk}
\usepackage[square,super]{natbib}
\usepackage{amsmath}
\usepackage{amsfonts}
\usepackage{bbm}
\usepackage{url}

\newcommand{\y}{\ensuremath{\mathbbm{y}}}

\title{\textbf{Coding theory in the streaming model}}
\author{Amit Kumar Sinhababu\thanks{amitks@cse.iitk.ac.in} }
\author{Vikraman Choudhury\thanks{vikraman@cse.iitk.ac.in} }
\affil{Department of Computer Science and Engineering,\\
  Indian Institute of Technology Kanpur
}

\begin{document}

\maketitle

\section{Introduction}
The theory of error correcting codes is a very important problem from both theoretical and practical points of view.
We want to look at this rich and developed theory in the data streaming model.

In a typical setting of data streaming, one is allowed to see the input in only a constant number
of one-way passes over it, and only has poly-log space in which to do intermediate calculations.
A sample coding theory problem in streaming setting would be as follows:
Can one recognize a Reed-Solomon codeword in one-pass using only poly-log space?

\section{Motivation}

\subsection{Applications in theoretical computer science}
Coding theory is used in several important results in theoretical computer science, notably
PCP theorem, de-randomization techniques, space lower bound proofs in data streaming, group testing problem in data streaming, pseudo-randomness.
We would like to explore coding theory from the point of view of data streaming, and also try to apply the concepts to other areas of theoretical computer science.

\subsection{Practical applications}

Optical storage (CDs and DVDs), RAID and ECC memory, network protocols, use simple codes
such as parity code and checksums which have trivial data stream algorithms for error detection.
However, both parity code and checksums are very limited in error detecting capabilities.
This is not acceptable in recent times of data explosion, as errors are becoming more frequent.

Reed-Solomon (RS) codes, are being increasingly used in storage systems (e.g. in
CDs and DVDs and more recently in RAID), and are well known to have good error
correcting capabilities. However, the conventional error detection for RS codes
is not space or pass efficient. Thus, a natural question to ask is if one can design
a data stream algorithm to perform error detection for RS codes.

In practical deployments of RS codes, size of input is relatively small.
However, if one needs to implement the error detection algorithm in controllers on disks
then it would be advantageous to use a data stream algorithm
so that it is feasible to perform error detection with every read.

A one-pass data streaming algorithm is also advantageous for data scrubbing since each bit will only be probed once,
which is good for longevity of data.

\section{Codeword testing\cite{sublinear}}

Let $C: \Sigma^k \to \Sigma^n$ be a code with distance $d$.
The general codeword testing problem is defined as follows:\\

\textbf{Input:} A vector $\y \in \Sigma^n$ and integer parameters $0 \le \tau_1 < \tau_2 \le n$.

\textbf{Output:} Decide whether $\Delta(\y,C) \le \tau_1$ or $\Delta(\y,C) \ge \tau_2$.\\

where $\Delta(\y,C)$ is the Hamming distance of \y\ from the closest codeword in $C$.

An ideal algorithm should solve the problem in one-pass, taking $log^{O(1)}n$ space, for some good code $C$.
The case $\tau_2 = \tau_1 + 1$ is already solved in one pass and space $\tilde{O}(\tau_1)$\cite{rudra2010data}.

It has been shown\cite{rudra2010data} that for the error detection problem,
there exists a randomized one-pass, log-space data stream algorithm for
a broad class of asymptotically good codes, including the RS and expander codes.
Deterministic data stream algorithms (even with multiple passes) require linear space for such codes\cite{rudra2010data}.

\subsection{Local and Tolerant Testing}
The local testing problem for a code $C \subseteq \Sigma^n$ is the following:\\

Given a received word $\y \in \Sigma^n$, we need to figure out if $\y \in C$ or
if \y\ differs from every codeword in $C$ in at least $0 < e \le n$ positions,
that is the Hamming distance of \y\ from every $c \in C$, is at least $e$.
If $e = 1$, then this is the error detection problem.\\

In the Tolerant testing problem, given \y\ we need to decide
if \y\ is at a distance at most $e_1$ from some codeword or
if it has distance at learning $e_2 > e_1$ from every codeword in $C$.

Both these problems reduce to the general codeword testing problem.
Some upper and lower bounds are shown by \citeauthor*{rudra2010data}\cite{rudra2010data}.

\subsection{Polynomial fitting}
Given a stream of $(x,y)$ points, the problem is to find univariate polynomials that
best fit the data. Over finite fields, this problem reduces to decoding RS codes,
while over the reals, it corresponds to the well studied polynomial regression problem.

\citeauthor*{mcgregor2011polynomial}\cite{mcgregor2011polynomial} shows one-pass algorithms for two natural problems:
\begin{itemize}
\item Find the polynomial of a given degree $k$ that minimizes the error
\item Find the polynomial of smallest degree that interpolates through the points with at most a given error bound
\end{itemize}

\section{Questions to be resolved}
\label{sec:problems}

\subsection{}
Solve the codeword testing problem with one pass and $\tilde{O}(min(k,\tau_1))$ space.
The very special case of the problem for $k = \tau_1 = \sqrt{n}$ with one pass and space $o(\sqrt{n})$ is also open.
This is open even for the special case of RS codes\cite{sublinear}.

Our main focus will be on this problem and improving the current results (if possible),
but we would also like to look at some related problems like the following.

\subsection{}
When can irreducible polynomials be detected in a streaming setting?\cite{uurtamo-proposal}

To be specific, what are the correct upper and lower bounds for space and passes required to detect
an irreducible degree $k$ polynomial over a finite field $\mathbb{F}_q$?
This is easy in one pass given $k\log{q}$ space.
\citeauthor*{uurtamo-proposal}\cite{uurtamo-proposal} suggests
``possible lower bounds could come from communication complexity –-
a first place to look might be at the counting function for irreducible polynomials''.

\subsection{}
Given an RS received word $r$ (that is, a transmitted codeword $c$ from a RS code $C$ added to an unknown error vector $e$,
with the Hamming weight of $e$ less than half the distance of $C$) and a target symbol $s$,
what is the correct (streaming) lower bound for evaluating $c$ at $s$?\cite{uurtamo-proposal}

\subsection{}
What are the correct upper and lower bounds for tolerant testing?\cite{uurtamo-proposal}

\subsection{}
As these problems are related to many other algebraic and combinatorial problems as well. Hence,
it would be interesting to look at those problems in data streaming model, and some related topics,
like property testing and Goldwasser's newly introduced concept of pseudo-deterministic algorithms\cite{DBLP:journals/eccc/GoldreichGR12}.
We recently came across a technique which approximates high degree polynomials using low degree rational functions.
It would be interesting if we can apply this trick in some data streaming problems.

\section{Papers to be surveyed}
\label{sec:papers}
\citet*{rudra2010data}\\
\citet*{mcgregor2011polynomial}\\
\citet*{guruswami2005tolerant}

\section{Tentative Timeline}
\begin{tabular}{ l | p{6cm} }
  Week 1 & Review basic concepts of coding theory, finite fields, communication complexity \\
  Weeks 2 - 3 & Survey literature (section~\ref{sec:papers}) \\
  Weeks 4 - 7 & Work on the problems (section~\ref{sec:problems})
\end{tabular}

\bibliographystyle{unsrtnat}
\bibliography{proposal}

\end{document}